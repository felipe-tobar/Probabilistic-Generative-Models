\chapter{Bayesian Nonparametrics}

\textbf{NB:} This is based on \cite[Ch.~1-2]{Orbanz2014BayesianNP} and \cite[Ch.~2]{rasmussen2006gpml}.

\section{Motivation}

Recall that, given a collection of observations $x_1,\ldots,x_n$, the main objective of the learning problem is to identify the data-generating mechanism. In classical statistics, the first step to achieve this is by defining the set of possible data generators as follows.  
\begin{definition}[Statistical model]
Let $\cX$ be a the (observation) sample space.  
A statistical model is a family of probability distributions
\[
\cP = \{ P_\theta : \theta \in \Theta \}
\]
defined on $\cX$, where $\Theta$ is a parameter space and $\theta$ is an unknown parameter indexing the data-generating distribution.
\end{definition}

Then, by considering the observations $x_1,\ldots,x_n$ as a realisation of a random variable with a distribution in the statistical model, learning reduces to the inverse problem related to identifying the distribution in $\cP$ that generated the observations. The canonical procedure to choose this model is maximum likelihood. 


\begin{example}[Density estimation: Gaussian parametric model vs.\ empirical measure]
Let $x_1,\dots,x_N$ be i.i.d.\ observations taking values in $\cX=\R$.

Consider the statistical model given by 
\[
\cP_{\mathrm{Gauss}}
= \bigl\{ P_{\mu,\sigma^2} : (\mu,\sigma^2)\in\R\times(0,\infty)\bigr\},
\qquad
p(x\mid \mu,\sigma^2)=\frac{1}{\sqrt{2\pi\sigma^2}}
\exp\!\Bigl(-\frac{(x-\mu)^2}{2\sigma^2}\Bigr).
\]
The optimal (maximum likelihood) parameters for this statistical model are
\[
\hat\mu=\frac{1}{n}\sum_{n=1}^N x_n,
\qquad
\widehat{\sigma^2}=\frac{1}{N}\sum_{n=1}^N (n_i-\hat\mu)^2.
\]
Alternatively, we can consider an assumption-free estimator of the data-generating distribution given by the empirical measure
\begin{equation}
    \hat P_N \;=\; \frac{1}{N}\sum_{n=1}^N \delta_{x_n},   
    \label{eq:empirical_measure_example}
\end{equation}
where $\delta_{x}$ is the Dirac measure at $x$. Fig.~\ref{fig:density_estimation_2methods} provides an illustration of both estimators. 

Note that for the Gaussian statistical model, the parameter space is clearly defined: $(\mu,\sigma^2)\in \Theta = \R\times(0,\infty)$. However, it may not be clear what the parameters are in the second case. From eq.~\eqref{eq:empirical_measure_example} notice that the parameters are the locations of the diracs, which are equal to the observations. This implies that the statistical model is the space of all measures with an arbitrary number of diracs in $\R$. This implies that the number of parameters in the second case is infinite. 
\end{example}

\begin{figure}
    \centering
    \includegraphics[width=0.7\textwidth]{img/week4_density_estimation_2methods.pdf}
    \caption{Density estimation: Gaussian model versus empirical measure.}
    \label{fig:density_estimation_2methods}
\end{figure}


 \begin{remark}
    We will call the statistical model $\cP$ \textbf{parametric} if the parameter space is finite, and \textbf{nonparametric} is parameter space is infinite.  
 \end{remark}

\paragraph{The role of the parameter.} We can consider learning the parameter as a form of compression, where the information in the, say $N$, observations is summarised in the parameter $\theta\in\Theta$---see Fig.~\ref{fig:data_model}. The parametric representation will then capture the relevant patterns in the data and retain them for subsequent tasks such as performing predictions, discriminative tasks such as classification, and even storage. Critically, when the parameter space is finite (and the dimensionality is lower than the number of datapoints), this compression is \emph{lossy}, meaning that some information in the data is inevitably lost when determining the parameter, and thus the complete observations cannot be fully recovered  (or explained) from the parameters alone. Additionally, and from an intuitive perspective, for a fixed number of parameters, an increasing amount of observations will not provide a monotonically-increasing amount of information, as there is only enough capacity in a $m$-dimensional parametric model.

On the contrary, a non-parametric model (i.e., a statistical model with an  infinite-dimensional parameter space) can represent a form of lossless data representation. This means that no information is lost when training the model. In such case, the distinction between the parameter space and the model space is usually blurred, as the parameter becomes the model itself. Though this sounds like a great way to learn models as its provide infinitely-flexible models that do not \emph{saturate} and keep learning as we feed more data to them, their implementation in practice requires careful considerations to deal with computational complexity. 


\begin{figure}
    \centering
    \includegraphics[width=0.7\textwidth]{img/week4_data-model.pdf}
    \caption{Illustration of parameter learning as compression.}
    \label{fig:data_model}
\end{figure}

\paragraph{Bayesian nonparametric models.} Since in Bayesian ML we consider the parameter to be a random variable, we need to define a (prior) distribution over the parameter space $\Theta$.  This 

\begin{definition}[Bayesian statistical model]
A Bayesian statistical model describes how data are generated, together with a way to encapsulate prior knowledge of the parameters before observing the data.

Formally, it consists of:
\begin{enumerate}
  \item a \emph{likelihood} $p(x \mid \theta)$, which specifies how the data $x$ are distributed for a given parameter value $\theta$, and
  \item a \emph{prior distribution} $\pi(\theta)$, which represents our beliefs about plausible values of $\theta$ before seeing any data.
\end{enumerate}

Together, the likelihood and the prior define a joint density
\[
p(x,\theta) = p(x \mid \theta)\,\pi(\theta),
\]
which describes both how the data are generated and how the parameters are distributed.
\end{definition}

Therefore, under the Bayesian paradigm, learning the model no longer refers to finding the best parameter in the parameter space (and, as a consequence the best model in the statistical model) as in the classical setup outlined above. Instead, we are now interested in finding the posterior distribution over the parameter 
\begin{equation}
    p(\theta|x_1,\ldots,x_N).
\end{equation}
As a consequence, the parameter remains uncertain given a finite number of observations, but this uncertainty is reduced as we see the observations.

This week we will focus on Bayesian models given by a infinite-dimensional parameter space, where the main challenge will need to define priors in such spaces.  Recall that a distribution on an infinite-dimensional space, or equivalently, a collection of random variables indexed by such space is a \emph{stochastic process} with paths in $\Theta$. In particular, we will focus on two such models: The Dirichlet process, which is a prior over probability distributions, and the Gaussian process which is a prior over functions. 

\warning{The Dirichlet process is not part of the 2025/26 version of this course. However, a brief (draft) presentation is kept in the lecture notes for completeness; note that this draft is not complete and might have some notational incosisntencies. Interested reader are recommended to see \cite{Orbanz2014BayesianNP} for a clear presentation of the Dirichlet process in ML.}
\section{The Dirichlet process}

\paragraph{Bayesian mixture model} We are very familiar at this stage with the hierarchical GMM, let us consider a different view of this model. Denoting the assignment variable as $z\in\N$, the distribution the observation $x$ generated by a the $k$-th cluster can be expressed as 
\begin{equation}
    p_k(x) = p(x \mid z=k),
\end{equation}
where we have not assumed that this distribution is Gaussian. Furthermore, the probability for a given observation to be generated by such cluster, can be denotes as 
\begin{equation}
    g_k = \Prob{Z=k},
\end{equation}
where $\sum_{\in\N}c_k = 1, c_k\geq0,\, \forall k\in\N$. The resulting model, given by
\begin{equation}
p(x) = \sum_{k\in\N} c_k p(x \mid z=k).    
\end{equation}
We will refer to this distribution as a mixture model. Furthermore, when there is only a finite number of probabilities $c_k$, we will say that this is a finite mixture. 

The set of all sequences of the the form $\{c_k\}_{k\in\N}$ is called \emph{simplex}, denoted by 
\begin{equation}
    \Delta \defeq \left\{ \{c_k\}_{k\in\N} \middle| \sum_{k\in\N}c_k = 1, c_k\geq0 \right\}.
\end{equation}
We can also assume that the cluster densities are in a parametric model over $\cX$, given by $\{p(\cdot \mid \phi) | \phi\in\Phi\}$. This way, the assignment variable informs the chosen parameters for the conditional probabilites. This yields the following expression for the generative model
\begin{equation}
    p(x) = \sum_{k\in\N} c_k p(x \mid \phi).    
\end{equation}
This allows us to express the following interpretation: Consider a discrete (atomic) probability measure $\theta$ on the parameter space $\Phi$, given by:
\begin{equation}
    \theta(\cdot) = \sum_{k\in\N} c_k \delta_{\phi_k}(\cdot). 
\end{equation}
This measure assigns probability to atoms in the parameter space and is referred to as the mixing measure---see Fig.~\ref{fig:mixing_model}. This is because we can express the PGM as
\begin{align}
    p(x) 
    &= \int_\Phi p(x \mid \phi) \theta(\phi)\d\phi\\
    &= \int_\Phi p(x \mid \phi) \sum_{k\in\N} c_k \delta_{\phi_k}(\phi)\d\phi\\
    &= \sum_{k\in\N}  c_k  p(x \mid \phi_k).
\end{align}

\begin{remark}
    The statistical model used in clustering can be written as the mixture model above. This implies that the parameter of such model is a discrete probability distribution (called the mixing distribution). 
\end{remark}


\begin{figure}
    \centering
    \includegraphics[width=0.7\textwidth]{img/week4_mixing_measure.pdf}
    \caption{Illustration of a mixing measure (left) and the mixing model (right).}
    \label{fig:mixing_model}
\end{figure}



Having identified the parameter space of the mixture model, let us now consider a Bayesian mixture model, that is, a random mixing measure 
\begin{equation}
    \Theta = \sum_{k\in\N} C_k \delta_{\Phi_k}.
\end{equation}
Therefore, the prior of a Bayesian mixture is the distribution of the random mixing law $\Theta$. Constructing this prior is not difficult. First, we can to define a prior for the parameters of the components, that is, 
\begin{equation}
    \Phi_1,\Phi_2,\ldots, \sim_\text{i.i.d.} G,
\end{equation}
and independent of $C_k$. Then to sample the weights we cannot consider indepenence, that is the collection elements of the sequence $\{c_k\}_{k\in\N}$ needs to add up to one. However, in the finite case, we can simply sample $K$ i.i.d.~elements $V_k$ form [0,1], and then define
\begin{equation}
    C_k \defeq \frac{V_x}{V_1+\ldots+V_K}.    
\end{equation}

\begin{remark}
    When the RVs $V_k$ follow a gamma distribution, $C_k$ follow a Dirichlet distribution. 
\end{remark}

\paragraph{Stick-breaking construction.} The above construction works well for finite $k$ but very often we will want to define the mixture over infinite components (Why?). For an infinite number of components, the sum of i.i.d.~variables $V_1 + V_2 + \ldots$ will diverge almost surely. 

Sampling an infinite number of parameters $\Phi_k\sim G$ is not problematic. For the mixing weights $C_k$, we can consider the stick-breaking construction, which provides a simple way to generate an infinite sequence of non-negative weights that sum to one.

Imagine a stick of length 1 representing total probability mass. The idea is to break this stick an infinite number of times, where the remaining pieces of the stick will be the $c_k$.

At step $k$, we:
\begin{enumerate}
  \item break off a fraction $V_k \in (0,1)$ of the remaining stick,
  \item assign this piece length $C_k$ to component $k$,
  \item keep the rest for future components.
\end{enumerate}

Formally, let $|I_1| = 1$. For $k = 1,2,\ldots$:
\[
V_k \sim H, \qquad
C_k = |I_k|\,V_k, \qquad
|I_{k+1}| = (1 - V_k)\,|I_k|.
\]

The weights $(C_k)_{k\ge1}$ are non-negative and satisfy
\[
\sum_{k=1}^\infty C_k = 1 \quad \text{almost surely}.
\]

This construction allows us to define mixture models with infinitely many components, while ensuring that the total probability mass remains finite and well defined.

\begin{figure}
    \centering
    \includegraphics[width=0.9\textwidth]{img/week4_stick_breaking.pdf}
    \caption{Illustration of a the stick breaking process.}
    \label{fig:stick_breaking}
\end{figure}

The strick-breaking procedure allows for constructing a mixing measure with infinite components by choosing a specific distribution for the components's parameters $G$. Choosing $H$ above as a beta distribution we have: 

\begin{definition}[Dirichlet process]
    Consider $\alpha>0$ a concentration parameter,  and $G$ a probability measure on $\Phi$. The random discrete probability measure $\Theta$ generated by 
    \begin{align}
        V_1,V_2,\ldots, &\sim_\text{i.i.d.} \text{Beta}(1,\alpha) \quad \text{and}\quad C_k \defeq V_k\prod_{j=1}^{k-1}(1-V_k)\\
        \Phi_1,\Phi_2,\ldots, &\sim_\text{i.i.d.} G
    \end{align}
    is called a Dirichlet process (DP) with base measure $G$ and concentration $\alpha$.
    
\end{definition}

Fig.~\ref{fig:2DPs} shows draws from two DPs with different concentration parameters. 

\begin{figure}
    \centering
    \includegraphics[width=0.9\textwidth]{img/week4_2DPs.pdf}
    \caption{Draws from two DPs with a standard Gaussian as base measure, and different concentration parameters.}
    \label{fig:2DPs}
\end{figure}





\paragraph{Dirichlet process mixture model.}
Using a Dirichlet process as a prior on the mixing measure yields an infinite mixture model.
Specifically, let
\[
\Theta \sim \mathrm{DP}(\alpha, G), \qquad
\phi_i \mid \Theta \sim \Theta, \qquad
x_i \mid \phi_i \sim p(x \mid \phi_i),
\]
for $i=1,\ldots,N$.
Marginally, this defines a mixture model with a countably infinite number of components, where the number of components effectively used by the data grows with $N$.

\paragraph{Posterior of the Dirichlet process.}
A key property of the Dirichlet process is conjugacy.
If
\[
\Theta \sim \mathrm{DP}(\alpha, G)
\quad\text{and}\quad
\phi_1,\ldots,\phi_N \mid \Theta \sim \Theta,
\]
then the posterior distribution of $\Theta$ is again a Dirichlet process:
\[
\Theta \mid \phi_1,\ldots,\phi_N
\;\sim\;
\mathrm{DP}\!\left(
\alpha + N,\;
\frac{\alpha G + \sum_{n=1}^N \delta_{\phi_n}}{\alpha + N}
\right).
\]

This result shows that Bayesian updating under the Dirichlet process corresponds to combining the prior base measure with the empirical distribution of the observed atoms.





\section{The Gaussian process}


Recall the parametric linear regression model given by
\[
y = \phi(x)^\top \w + \varepsilon,
\qquad
\varepsilon \sim \cN(0,\sigma^2),
\]
where $x\in\R^d$, $\phi: \R^d \to\R^m$ is a fixed feature map, and $\w\in\R^m$ is the unknown parameter. 

Let us consider a set of observation inputs $x_1,\ldots,x_N$, and denote the vector representation for the input, features and corresponding outputs: 
\[
\X=\begin{bmatrix}
x_1^\top\\
x_2^\top\\
\vdots\\
x_N^\top
\end{bmatrix} \in\R^{N\times d},
\qquad
\Phi \coloneqq
\begin{bmatrix}
\phi(x_1)^\top\\
\phi(x_2)^\top\\
\vdots\\
\phi(x_N)^\top
\end{bmatrix}\in\mathbb{R}^{N\times m}
\qquad
\y=\begin{bmatrix}
y_1\\
y_2\\
\vdots\\
y_N
\end{bmatrix} \in\R^{N}.
\]

Given the Gaussian driving noise $\varepsilon$, the observations $\y$ are distributed according to 
\begin{equation}
    p(\y\mid \X,\w) = \cN(\Phi\w,\sigma^2I_N).
    \label{eq:Bayesian_linear_model}
\end{equation}


\begin{remark}
    For a fixed map $\phi$, the parameter space of this statistical model is $\{(w,\sigma) \in\R^m\times\R_+\}$.
\end{remark}


From a Bayesian standpoint, let us consider a standard Gaussian prior on the weights,
\begin{equation}
\w \sim \mathcal{N}(0,\Sigma_\w),
 \label{eq:Bayesian_linear_model_prior}
\end{equation}
with $\Sigma_\w$ a positive definite matrix.

This is a standard latent-variable formulation as the models we have seen earlier. In fact, the marginal law of $\y$, given by 
\begin{equation}
    p(y \mid \X) = \int  p(y \mid \X, \w)p(\w)\d\w,
    \label{eq:marginal_implicit}
\end{equation}
can be calculated analytically: eqs.~\eqref{eq:Bayesian_linear_model} and  \eqref{eq:Bayesian_linear_model_prior} into  \eqref{eq:marginal_implicit} give

\[
p(\y\mid \X)
= \mathcal N\!\bigl(\y \,\big|\, 0,\; \Phi\Sigma_{\w}\Phi^\top+\sigma^2 I_N\bigr).
\]


\begin{remark}
    Due to the linear-Gaussian formulation of this model, we say that the (latent) parameter can be marginalised out (or integrated out) analytically.
\end{remark}


Furthermore, by conjugacy, the posterior distribution of the weights is Gaussian,
\[
p(\w\mid \y,\X)=\mathcal N(\w\mid \mu,\Sigma),
\]
with
\[
\Sigma=\Bigl(\Sigma_{\w}^{-1}+\frac{1}{\sigma^2}\Phi^\top\Phi\Bigr)^{-1},
\qquad
\mu=\Sigma\Bigl(\frac{1}{\sigma^2}\Phi^\top \y\Bigr).
\]


\begin{remark}

Observe that the posterior mean of $\w$ satisfies
\[
\mu \defeq
\mathbb E[\w\mid \y,\X]= \argmin_{\w}
\left\{
\frac{1}{2\sigma^2}\|\y-\Phi\w\|_2^2
+\frac{1}{2}\w^\top \Sigma_{\w}^{-1}\w
\right\},
\]
meaning that \(\mathbb E[\w\mid \y,\X]\) coincides with the ridge regression estimator (aka $L_2$-regularised least squares). In  particular, when using an uninformative prior $p(\w)$, the solution reduces to the ordinary least squares estimator.
\end{remark}





\begin{example}
    Consider a \emph{localised} version of this model where $x\in\R$, and the features $\phi$ are Gaussian bumps, i.e., $\phi_i = \exp\left(-\gamma(x-c_i)^2\right)$, with $\gamma$ and $c_i$ fixed hyperparameters.  Fig.~\ref{fig:Bayes_linear} shows four realisations of the model to illustrate its flexibility. 

    \begin{figure}[t]
    \centering
    \includegraphics[width=0.9\textwidth]{img/week4_Bayesian_linear_model.pdf}
    \caption{4 realisations for the Bayesian linear model using Gaussian features.}
    \label{fig:Bayes_linear}
\end{figure}

\end{example}

Observe that the posterior predictive distribution for the latent function value 
\(f_\ast=\phi(x_\ast)^\top\w\), 
at a new test input $x_\ast$ satisfies,
\[
p(f_\ast \mid x_\ast, \X, \y)
= \int p(f_\ast \mid x_\ast, \w)\, p(\w \mid \X, \y)\,\mathrm d\w .
\]
To calculate this expression, recall that
\[
f_\ast \mid x_\ast, \w = \phi(x_\ast)^\top \w,
\qquad
\w \mid \X, \y \sim \mathcal N(\mu,\Sigma),
\]
which allows to calculate the integral in closed form, yielding a Gaussian predictive distribution,
\[
p(f_\ast \mid x_\ast, \X, \y)
= \mathcal N\!\bigl(f_\ast \,\big|\,
  m_\ast,\;
v_\ast
\bigr),
\]
with the mean and variance given by 
\begin{align}
    m_\ast 
&= \phi(x_\ast)^\top \mu
=\phi_\ast^\top \Sigma_{\w}\Phi^\top K^{-1}\y,\\
v_\ast
&= \phi(x_\ast)^\top \Sigma \,\phi(x_\ast)
=\phi_\ast^\top \Sigma_{\w}\phi_\ast
\;-\;
\phi_\ast^\top \Sigma_{\w}\Phi^\top K^{-1}\Phi\Sigma_{\w}\phi_\ast,
\end{align}
where we have denoted \(\phi_\ast \coloneqq \phi(x_\ast)\in\mathbb{R}^m\) and
\[
K \;\coloneqq\; \Phi \Sigma_{\w}\Phi^\top + \sigma^2 I_N \in\R^{N\times N}.
\]


\section{Construction of the GP}

\begin{definition}[Gaussian process]
A \emph{Gaussian process} (GP) is a stochastic process
\(\{f(x) : x \in \mathcal X\}\) such that for any finite set of inputs
\(x_1,\ldots,x_N \in \mathcal X\), the random vector
\[
\bigl[f(x_1),\ldots,f(x_N)\bigr]\in\R^N
\]
is multivariate Gaussian. A GP is fully specified by a mean function
\(m:\mathcal X\to\mathbb R\) and a covariance (kernel) function
\(k:\mathcal X\times\mathcal X\to\mathbb R\), and denoted by 
\[
f \sim \GP{m(\cdot)}{k(\cdot,\cdot)}.
\]
This means that for any \(x_1,\ldots,x_N\),
\[
\bigl[f(x_1),\ldots,f(x_N)\bigr]
\sim
\mathcal N\!\left(
\bigl[m(x_1),\ldots,m(x_N)\bigr],
\;
\bigl[k(x_i,x_j)\bigr]_{i,j=1}^N
\right).
\]    
\end{definition}

The model above, defined by \(f(x)\coloneqq \phi(x)^\top \w\) with the  prior \(\w\sim\mathcal N(0,\Sigma_{\w})\), is a GP with
\[
m(x)=0,
\qquad
k(x,x')=\phi(x)^\top \Sigma_{\w}\,\phi(x').
\]

\begin{remark}[Zero-mean GPs]
    We will consider zero-mean GPs only, since the feature map $\phi$ can be extended with a constant coordinate, that is $\tilde\phi^\top = [c, \phi^\top], c\in\R$, which produces a constant term carrying the first coordinate of the parameter $w_1c$. Therefore, this term can be considered as a trainable mean.
\end{remark}
Furthermore, considering the observation noise \(\varepsilon\sim\mathcal N(0,\sigma^2)\), we have that $y$ is also a GP:
\[
y(\cdot)\sim \GP{0}{k(x,x')+\sigma^2\mathbf{1}\{x=x'\}}.
\]

For an input-output observation dataset \(\{\X,\y\}\), define \(K= \Phi\Sigma_{\w}\Phi^\top \in\mathbb R^{N\times N}\), i.e.
\[
K_{ij}=k(x_i,x_j)=\phi(x_i)^\top \Sigma_{\w}\,\phi(x_j),
\qquad i,j\in\{1,\ldots,N\}.
\]
Then the (GP) likelihood admits the explicit density
\begin{equation}
    p(\y\mid \X)
=
\frac{1}{(2\pi)^{N/2}\,|K+\sigma^2 I_N|^{1/2}}
\exp\!\left(
-\frac{1}{2}\,\y^\top (K+\sigma^2 I_N)^{-1}\y
\right).    
\label{eq:GP_likelihood}
\end{equation}

\begin{remark}[Inducing similarties through $\phi$]
The covariance of the GP reveals a \emph{similarity-based structure} controlled by inner products between the chosen feature map \(\phi\) (rotated by \(\Sigma_{\w}\)). Therefore, different notions of similarity can be induced by appropriate choices of \(\phi\). 
For instance, taking \(\Sigma_{\w}=I\) and \(\phi(x)=x/\|x\|\) yields a covariance that is maximised when the inputs are colinear. 
This motivates the use of high-dimensional (and possibly trainable) feature maps \(\phi\) to extract representations that are relevant for the task.
\end{remark}

\begin{remark}[Cubic computational cost]
Training and inference under the GP model, i.e. evaluating the marginal likelihood and making predictions, requires inversion of the \(N\times N\) covariance matrix \(K+\sigma^2 I_N\). 
As a consequence, the computational complexity scales as \(\mathcal O(N^3)\) and depends on the number of data points \(N\), but not explicitly on the feature dimension \(m\).
\end{remark}

\begin{remark}[Evidence for the kernel trick]
Although the model may be expressed in terms of feature maps (which we would like to be high-dimensional) all interactions between data points occur exclusively through inner products
\(\phi(x_i)^\top \Sigma_{\w}\phi(x_j)\).
Hence, individual features never appear alone but only through their pairwise similarities. This observation calls for exploiting the kernel trick to represent the model.
\end{remark}

\begin{remark}[Avoiding degeneracy in GPs]
Considering \(m\) basis functions (the dimension of $\phi$) and \(N>m\) observations, the covariance matrix \(K=\Phi\Sigma_{\w}\Phi^\top\) is rank-deficient.
In this case, the Gaussian likelihood is degenerate in the noiseless case, and the GP does not have a density.
While the addition of observation noise \(\sigma^2 I_N\) ensures invertibility, a well-posed GP that admits a large number of observations \(N\) without degeneracy requires feature maps \(\phi\) of (effectively) infinite dimension.
This motivates infinite-dimensional feature representations.
\end{remark}

To construct a GP with an infinite-dimensional feature map and use it in practice, the explicit treatment of such a map should be bypassed altogether. This is feasible, since the likelihood of the GP, defined in eq.~\eqref{eq:GP_likelihood}, only depends on the map via the covariance entries $k(x_i,x_j)=\phi(x_i)^\top \Sigma_{\w}\,\phi(x_j)$. 

\begin{remark}[The kernel trick]
Specifying a GP directly by choosing a kernel function \(k\), rather than an explicit feature map \(\phi\), requires  choosing \(k\) to be positive definite.
By \emph{Mercer’s theorem}, any continuous, symmetric, positive definite kernel admits a (possibly infinite-dimensional) feature representation
\[
k(x,x')=\langle \phi(x),\phi(x')\rangle_{\mathcal H},
\]
for some Hilbert space \(\mathcal H\).
Thus, working directly with positive-definite kernels implicitly defines a corresponding feature map without ever constructing it explicitly.
\end{remark}


\section{Implementing a GP}

\paragraph{Choosing the kernel.}


\begin{remark}[Kernel as a similarity measure]
    Recall the the kernel defines the inner product between the feature maps, as a consequence, its choice will be made on the basis of assessing similarity.
\end{remark}


The usual choice for a GP kernel is the \textit{Squared Exponential} (SE) kernel given by 
\begin{equation}\label{eq:gp_kernel_se}
	k_{SE}(x, x') = \sigma^2 \exp\left( - \frac{\left( x- x'\right)^2}{2\ell^2} \right).
\end{equation}

\begin{remark}
    If the kernel is a function of the difference of its arguments, that is, $k(x, x')=k(x-x')$, we will say that that the kernel is stationary. It is worth noting that stationary kernels make the covariance between points invariant under translations in the input space. An important notion is that a kernel can be viewed as a measure of similarity between points, and in the case of stationary kernels, the closer two points are, the more similar they are.
\end{remark}

The \emph{Rational Quadratic} (RQ) kernel given by 
 \begin{equation}\label{eq:gp_kernel_rq}
	k_{RQ}(x, x') = \sigma^2 \left(1 + \frac{\left( x- x'\right)^2}{2\alpha\ell^2 } \right)^{-\alpha}
\end{equation}
can be interpreted as the sum of infinite SE kernels with different \textit{lenghtscales}, where $\alpha$ controls the variablity of this scale parameter.

The periodic kernel given by 
\begin{equation}\label{eq:gp_kernel_p}
	K_{P}(x, x') = \sigma^2 \exp\left(-\frac{2\sin^2\left(\pi |x- x'| / p \right)}{\ell^2 } \right)
\end{equation}
allows for modelling periodic funcitons, where the parameter $p$ controls the period of the functions. An extension of this kernel is the locally-periodic kernel 
\begin{equation}\label{eq:gp_kernel_lp}
	K_{LP}(x, x') = \sigma^2  \exp\left(-\frac{\left(x- x' \right)^2}{2\ell^2 } \right) \exp\left(-\frac{2\sin^2\left(\pi |x- x'| / p \right)}{\ell^2 } \right)
\end{equation}
given by a periodic kernel mutiplied by an SE kernel, which allows for local periodic behaviour 

\begin{remark}
When designing a Gaussian process and choosing a covariance function, one is not restricted to a fixed set of known kernels.
Kernels can be combined to construct richer covariance structures that better capture the properties of the underlying process.
In particular, the sum and the product of valid kernels are themselves valid kernels, and the exponential of a kernel,
\(\exp(k_1(\cdot,\cdot))\) with \(k_1\) a valid kernel, also defines a valid covariance function.
These closure properties allow for flexible and expressive kernel design.
\end{remark}

\begin{figure}[t]
    \centering
    \includegraphics[width=0.9\textwidth]{img/week4_GP_kernel_and_samples.pdf}
    \caption{Covariance kernels (left) and GP samples (right). From top to bottom: Square exponential, rational quadratic, periodic, and locally-periodic kernels.}
    \label{fig:GP_samples}
\end{figure}


Fig.~\ref{fig:GP_samples} shows these four kernels and samples drawn from them.

\paragraph{Sampling from a Gaussian process.}

Given that a sample from a GP is an infinite-dimensional function, sampling form it might seem impossible at first. However, recall that a GP defines a joint Gaussian distribution over the function values \emph{at any finite collection of input points}. See Fig.~\ref{fig:gp_graphical_model_full} for a graphical model representation.


\begin{figure}[H]
\centering
\begin{tikzpicture}[node distance=1.6cm]

    
% Inputs (constants)
\node[const] (x) {$x_n$};

% Latent function values
\node[latent, right=of x] (f) {$f_n$};

% Observations
\node[obs, below=of f] (y) {$y_n$};

% Hyperparameters
\node[latent, right=of f] (theta) {$\theta$};
\node[latent, right=of y] (sig) {$\sigma^2$};

% Edges
\edge{x}{f}
\edge{theta}{f}
\edge{f}{y}
\edge{sig}{y}

% Plate
\plate{plateN}{(x)(f)(y)}{$n=1,\ldots,N$}


\end{tikzpicture}
\caption{Graphical model of a GP model over a finite collection of inputs.
The latent function \(f\) is drawn from a GP prior and evaluated at inputs \(x_n\), producing latent values \(f_n\), from which noisy observations \(y_n\) are generated.}
\label{fig:gp_graphical_model_full}
\end{figure}


Therefore, to sample a function from a GP, one first selects a finite set of input locations, computes the corresponding mean vector and covariance matrix, and then draws a sample from the resulting multivariate normal distribution. 


Each such draw corresponds to one possible realisation of the random function consistent with the specified mean and covariance. See Algorithm \ref{alg:gp_sampling}, which uses the Cholesky decomposition..

\begin{algorithm}[H]
\caption{Sampling from a Gaussian Process}
\label{alg:gp_sampling}
\begin{algorithmic}[1]
\Require Mean function \(m(\cdot)\), kernel \(k(\cdot,\cdot)\), inputs \(x_1,\ldots,x_N\)
\Ensure Sample \(\f = [f(x_1),\ldots,f(x_N)]^\top\)

\State Compute \(\m = [m(x_1),\ldots,m(x_N)]^\top\)
\State Compute \(K\in\mathbb R^{N\times N}\) with \(K_{ij}=k(x_i,x_j)\)
\State Draw \(\z \sim \mathcal N(0,I_N)\)
\State Compute \(L\) such that \(K=LL^\top\) (e.g.\ Cholesky)
\State Return \(\f = \m + L\z\)

\end{algorithmic}
\end{algorithm}

Then \(\f = [f(x_1),\ldots,f(x_N)]^\top\) is a sample from the Gaussian process evaluated at the chosen inputs.

\paragraph{Training a GP}

Training a Gaussian process amounts to learning the kernel hyperparameters
\(\theta\) (and the noise variance \(\sigma^2\)) from data.
Thanks to the marginalisation property of Gaussian processes, the (infinite-dimensional) latent
function $f$ can be integrated out analytically, yielding the
marginal likelihood
\[
p(\y \mid \X,\theta,\sigma^2)
=
\mathcal N\!\bigl(\y \,\big|\, 0,\; K_\theta+\sigma^2 I_N\bigr),
\]
where \((K_\theta)_{ij}=k_\theta(x_i,x_j)\).
Training is then formulated as the optimisation problem
\[
\max_{\theta,\sigma^2}\;
\log p(\y \mid \X,\theta,\sigma^2)
=
-\frac{1}{2}\y^\top (K_\theta+\sigma^2 I_N)^{-1}\y
-\frac{1}{2}\log|K_\theta+\sigma^2 I_N|
-\frac{N}{2}\log(2\pi),
\]
which balances data fit and model complexity and can be solved using
gradient-based methods.


\begin{algorithm}[H]
\caption{Training a Gaussian Process via Marginal Likelihood}
\label{alg:gp_training}
\begin{algorithmic}[1]
\Require Data \(\{(x_n,y_n)\}_{n=1}^N\), kernel family \(k_\theta(\cdot,\cdot)\), initial \(\theta,\sigma^2\)
\Ensure Trained hyperparameters \(\theta^\star,\sigma^{2\star}\)

\State Form \(\X=[x_1,\ldots,x_N]\) and \(\y=[y_1,\ldots,y_N]^\top\)
\Repeat
    \State Compute the kernel matrix \(K_\theta \in \R^{N\times N}\) with \((K_\theta)_{ij}=k_\theta(x_i,x_j)\)
    \State Set \(C \gets K_\theta + \sigma^2 I_N\)
    \State Evaluate the (log) marginal likelihood
    \[
    \ell(\theta,\sigma^2) \gets \log p(\y\mid \X,\theta,\sigma^2)
    = -\frac{1}{2}\y^\top C^{-1}\y - \frac{1}{2}\log|C| - \frac{N}{2}\log(2\pi)
    \]
    \State Compute gradients \(\nabla_\theta \ell(\theta,\sigma^2)\) and \(\partial_{\sigma^2}\ell(\theta,\sigma^2)\)
    \State Update \(\theta,\sigma^2\) using a gradient-based optimiser (e.g.\ gradient ascent)
\Until{convergence}
\State \Return \(\theta^\star \gets \theta,\ \sigma^{2\star} \gets \sigma^2\)

\end{algorithmic}
\end{algorithm}


\paragraph{Posterior computation} After training, posterior inference in Gaussian process regression is also made tractable by the marginalisation property. Conditioning the joint Gaussian distribution of the training set \(\X, \y\) and new inputs \(x_\ast\) yields a closed-form posterior predictive distribution.

Specifically, for a test input \(x_\ast\), the posterior over the latent function value \(f_\ast\) is Gaussian,
\[
p(f_\ast \mid x_\ast,\X,\y)
= \mathcal N\!\bigl(f_\ast \,\big|\, m_\ast,\; v_\ast \bigr),
\]
with mean and variance given by
\[
m_\ast = k_\ast^\top (K+\sigma^2 I_N)^{-1}\y,
\qquad
v_\ast = k(x_\ast,x_\ast) - k_\ast^\top (K+\sigma^2 I_N)^{-1}k_\ast,
\]
where \(k_\ast = [k(x_1,x_\ast),\ldots,k(x_N,x_\ast)]^\top\).
Thus, the posterior combines prior uncertainty encoded by the kernel with information from the data, and both prediction and uncertainty quantification reduce to linear algebra operations involving the kernel matrix.


\begin{algorithm}[H]
\caption{Posterior Prediction in Gaussian Process Regression}
\label{alg:gp_posterior}
\begin{algorithmic}[1]
\Require Training data \(\{(x_n,y_n)\}_{n=1}^N\), kernel \(k(\cdot,\cdot)\), noise variance \(\sigma^2\), test inputs \(x^\ast_1,\ldots,x^\ast_M\)
\Ensure Posterior predictive mean \(\m_\ast\in\mathbb R^M\) and covariance \(\Sigma_\ast\in\mathbb R^{M\times M}\) for \(\f_\ast=[f(x^\ast_1),\ldots,f(x^\ast_M)]^\top\)

\State Form \(\y=[y_1,\ldots,y_N]^\top\)
\State Compute \(K\in\mathbb R^{N\times N}\) with \(K_{ij}=k(x_i,x_j)\)
\State Compute \(K_\ast\in\mathbb R^{N\times M}\) with \((K_\ast)_{i\ell}=k(x_i,x^\ast_\ell)\)
\State Compute \(K_{\ast\ast}\in\mathbb R^{M\times M}\) with \((K_{\ast\ast})_{\ell r}=k(x^\ast_\ell,x^\ast_r)\)
\State Set \(C \gets K+\sigma^2 I_N\) and compute its Cholesky factorisation \(C=LL^\top\)
\State Solve \(L\mathbf{v}=\y\) and \(L^\top \alpha=\mathbf{v}\) \hfill (so \(\alpha=C^{-1}\y\))
\State Compute posterior mean \(\m_\ast \gets K_\ast^\top \alpha\)
\State Solve \(LW=K_\ast\) for \(W\) \hfill (so \(W=L^{-1}K_\ast\))
\State Compute posterior covariance \(\Sigma_\ast \gets K_{\ast\ast}-W^\top W\)
\State \Return \(\m_\ast,\Sigma_\ast\)
\end{algorithmic}
\end{algorithm}

\paragraph{Plotting a GP.} To maximise visual understanding, GPs are plotted in the following manner: The mean is plotted as a thick solid line, its ($\pm$ 2 std.~dev.) credible interval is represented by a shaded area, samples are plotted as thin lines, and---if present---observations are plotted as disconnected dots. . Fig.~\ref{fig:GP_full_example} provides plots for a complete example of the training and inference stage of a GP, including: samples from the prior, observed data, and the GP posterior with untrained and trained hyperparameters. The code for these figures is available in \url{https://github.com/felipe-tobar/GP-lite} 
\begin{figure}[t]
    \centering
    \includegraphics[width=0.9\textwidth]{img/week4_gp_prior.pdf}
    \includegraphics[width=0.9\textwidth]{img/week4_GP_samples.pdf}
    \includegraphics[width=0.9\textwidth]{img/week4_gp_posterior_untrained.pdf}
    \includegraphics[width=0.9\textwidth]{img/week4_gp_posterior_trained.pdf}
    \caption{Plots fro the full GP pipeline. From top to botom: prior, data, posterior predictive (untrained) and posterior predictive (trained).}
    \label{fig:GP_full_example}
\end{figure}
